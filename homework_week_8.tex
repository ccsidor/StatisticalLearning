\documentclass[11pt]{article}
\usepackage[pdftex]{graphicx}
\usepackage{float}
\usepackage[verbose]{wrapfig}
\usepackage[tbtags]{amsmath}
\usepackage{amsmath,bm}
\usepackage{amssymb}
\usepackage[square, comma, sort&compress]{natbib}
\usepackage{url}
\usepackage{color}
\usepackage{setspace}
\usepackage{verbatim}
\pagestyle{plain}
\pagenumbering{arabic}
\usepackage[font=small,labelformat=empty,labelsep=none]{caption}
\usepackage{todonotes}
\oddsidemargin 0in \evensidemargin 0in
\topmargin -21pt \headsep 10pt
\textheight 9.1in \textwidth 6.5in
\brokenpenalty=10000
\renewcommand{\baselinestretch}{1.0}
\bibpunct{[}{]}{,}{s}{}{;}
\bibliographystyle{unsrtnat}
\usepackage{datetime}


\begin{document}

\begin{center}
{\bf{ \large Homework Week 8.}}\\
\makeatletter\@date\makeatother
\end{center}

\section{Exercise 1}
To provide the student with exposure to Tensorflow we will provide the MNIST computer vision/ handwriting dataset which allows the training of a Neural Network for handwriting recognition. MNIST consists of images of handwritten digits, for example:
\begin{figure}[H]
\center
\includegraphics[width=10cm]{numbers.png}
\caption{Example images from MINST dataset.}
\label{minst}
\end{figure}

The student will use the provided basic software to train the provided Neural Network and then predict the results of the following images:
\begin{figure}[H]
\center
\includegraphics[width=10cm]{alternative_digits.png}
\caption{Alternative digit set for prediction.}
\label{alt}
\end{figure}

\section{Exercise 2}
Follow the same process as in Exercise 1 but using a sample of the students own handwriting. Provide results for the accuracy of prediction for the new dataset and the images provided above.

\section{What to hand in}
Provide a Jupyter notebook containing the training and predictions, for both datasets, based on the provided Jupyter notebook.
\end{document}

 
