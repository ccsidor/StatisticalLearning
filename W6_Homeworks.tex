\documentclass[a4paper]{article}

%% Language and font encodings
\usepackage[english]{babel}
\usepackage[utf8x]{inputenc}
\usepackage[T1]{fontenc}

%% Sets page size and margins
\usepackage[a4paper,top=3cm,bottom=2cm,left=3cm,right=3cm,marginparwidth=1.75cm]{geometry}

% packages
\usepackage{amsmath, amsfonts}
\usepackage{graphicx, multicol}
\usepackage{textgreek,mathrsfs,bbm,relsize,multirow,float}
\usepackage{amssymb,amsthm,mathtools,scrextend,stackengine}
\usepackage[table,xcdraw]{xcolor} % clashes with tikz package
\usepackage[colorinlistoftodos]{todonotes}
\usepackage{wrapfig}
\newenvironment{frcseries}{\fontfamily{frc}\selectfont}{}
%% commands
\newcommand{\xbar}{\mbox{\larger$\bar{x}$}}
\renewcommand{\baselinestretch}{1.5} %%1.5 spacing
\newcommand{\textfrc}[1]{{\frcseries#1}}
% commands

\newenvironment{sbmatrix}[1]
 {\def\mysubscript{#1}\mathop\bgroup\begin{bmatrix}}
 {\end{bmatrix}\egroup_{\textstyle\mathstrut\mysubscript}}
 
\pgfdeclarelayer{background}
\pgfsetlayers{background,main}

\begin{document}
\title{Homework Week 6}

\maketitle

\section{Exercise 1}

Consider how the different ensembles (bagging, boosting, and random subspace learning) work. Using the Ionosphere data set from the mlbench package, 50 replications, and a 60\%-training/40\%-test data split to compute the average test error, create a script for the following:

\begin{itemize}
\item Run regular LDA and C-SVM
\item Run boosted LDA and C-SVM
\item Run bagged LDA and C-SVM
\item Run RSSL LDA and C-SVM
\item Generate comparative boxplots showing the test errors for logistic regression  the three ensemble versions (total of 4 boxplots).
\item Compare the ensembles and argue which is best and why.
\end{itemize}

\textbf{What to hand in: Script to do the above, boxplot output, written argument.}

\section{Exercise 2}

\begin{itemize}
\item Create a scatterplot of the data and color the points according to the label
\item Use a stratified hold out split of the data into 75\% Training and 25\% Test, 50 replications, and compute the test error for LDA and QDA.
\item Generate comparative boxplots showing test error.
\item Create colorful scatterplots of the predicted labels from both LDA and QDA
\item Comment on the differences LDA and QDA produce and argue which is better for the data.
\end{itemize}

\textbf{What to hand in: Script to do the above, boxplot output, scatterplot outputs, written argument.}

\end{document}