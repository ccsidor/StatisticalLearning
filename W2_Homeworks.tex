\documentclass[a4paper]{article}

%% Language and font encodings
\usepackage[english]{babel}
\usepackage[utf8x]{inputenc}
\usepackage[T1]{fontenc}

%% Sets page size and margins
\usepackage[a4paper,top=3cm,bottom=2cm,left=3cm,right=3cm,marginparwidth=1.75cm]{geometry}

% packages
\usepackage{amsmath, amsfonts}
\usepackage{graphicx, multicol}
\usepackage{textgreek,mathrsfs,bbm,relsize,multirow,float}
\usepackage{amssymb,amsthm,mathtools,scrextend,stackengine}
\usepackage[table,xcdraw]{xcolor} % clashes with tikz package
\usepackage[colorinlistoftodos]{todonotes}
\usepackage{wrapfig}
\newenvironment{frcseries}{\fontfamily{frc}\selectfont}{}
%% commands
\newcommand{\xbar}{\mbox{\larger$\bar{x}$}}
\renewcommand{\baselinestretch}{1.5} %%1.5 spacing
\newcommand{\textfrc}[1]{{\frcseries#1}}
% commands

\newenvironment{sbmatrix}[1]
 {\def\mysubscript{#1}\mathop\bgroup\begin{bmatrix}}
 {\end{bmatrix}\egroup_{\textstyle\mathstrut\mysubscript}}
 
\pgfdeclarelayer{background}
\pgfsetlayers{background,main}

\begin{document}
\title{Homework Week 2}

\maketitle

\section{Exercise 1}
Consider the ORL-2 data set. There are 400 images of faces where each image has 648 dimensions (28 rows and 23 columns). Create a script to do the following:
\begin{itemize}
\item Select 9 faces at random (using a uniform distribution). 
\item Create code to evenly divide up each image into a 4 by 4 cross section so that the image consists of 16 equally large chunks total (hint: consider each image as a separate matrix, this would be like creating 16 partitions of that image matrix)
\item Using a uniform distribution, randomly select a chunk of the image
\item Randomly paint the chunk either black or white. This can be accomplished like flipping a coin or rolling a die where heads or evens are white and tails or odds are black.
\end{itemize}

\textbf{What to hand in: Script to do the above and an output of the 9 painted faces. }


\section{Exercise 2}
Now consider the 10 audio tracks of Crosby howling. Each track is 20 seconds and consists of Crosby howling while Jessica says 'Wow' very loudly. Create a script for the following:

\begin{itemize}
\item Read in each .wav file to R properly
\item Store all tracks into one matrix (truncating if necessary to make all tracks equal length)
\item Create code to evenly divide up each sound clip into 5 chunks of equal length.
\item Uniformly select one chunk per sound clip.
\item Again like flipping a coin, either replace the chunk with silence or constant loudness.
\item Compute the distance matrix of the sound matrix using your preferred distance measure, display a visual of the distance matrix, and comment on the peaks and lows.
\end{itemize}

\textbf{What to hand in: Script to do the above, distance matrix, visual distance matrix output, comments on distance matrix meaning.}

\end{document}
